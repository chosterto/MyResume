\documentclass[10pt, letterpaper]{article}

% Packages:
\usepackage[
    ignoreheadfoot, % set margins without considering header and footer
    top=1.5 cm, % seperation between body and page edge from the top
    bottom=1.5 cm, % seperation between body and page edge from the bottom
    left=2 cm, % seperation between body and page edge from the left
    right=2 cm, % seperation between body and page edge from the right
    footskip=1.0 cm, % seperation between body and footer
    % showframe % for debugging 
]{geometry} % for adjusting page geometry
\usepackage{titlesec} % for customizing section titles
\usepackage{tabularx} % for making tables with fixed width columns
\usepackage{array} % tabularx requires this
\usepackage[dvipsnames]{xcolor} % for coloring text
\definecolor{primaryColor}{RGB}{0, 0, 0} % define primary color
\usepackage{enumitem} % for customizing lists
\usepackage{fontawesome5} % for using icons
\usepackage{amsmath} % for math
\usepackage[
    pdftitle={John Cho's CV},
    pdfauthor={John Cho},
    pdfcreator={LaTeX with RenderCV},
    colorlinks=true,
    urlcolor=primaryColor
]{hyperref} % for links, metadata and bookmarks
\usepackage[pscoord]{eso-pic} % for floating text on the page
\usepackage{calc} % for calculating lengths
\usepackage{bookmark} % for bookmarks
\usepackage{lastpage} % for getting the total number of pages
\usepackage{changepage} % for one column entries (adjustwidth environment)
\usepackage{paracol} % for two and three column entries
\usepackage{ifthen} % for conditional statements
\usepackage{needspace} % for avoiding page brake right after the section title
\usepackage{iftex} % check if engine is pdflatex, xetex or luatex

% Ensure that generate pdf is machine readable/ATS parsable:
\ifPDFTeX
    \input{glyphtounicode}
    \pdfgentounicode=1
    \usepackage[T1]{fontenc}
    \usepackage[utf8]{inputenc}
    \usepackage{lmodern}
\fi

\usepackage{charter}

% Some settings:
\raggedright
\AtBeginEnvironment{adjustwidth}{\partopsep0pt} % remove space before adjustwidth environment
\pagestyle{empty} % no header or footer
\setcounter{secnumdepth}{0} % no section numbering
\setlength{\parindent}{0pt} % no indentation
\setlength{\topskip}{0pt} % no top skip
\setlength{\columnsep}{0.15cm} % set column seperation
\pagenumbering{gobble} % no page numbering

\titleformat{\section}{\needspace{4\baselineskip}\bfseries\large}{}{0pt}{}[\vspace{1pt}\titlerule]

\titlespacing{\section}{
    % left space:
    -1pt
}{
    % top space:
    0.3 cm
}{
    % bottom space:
    0.2 cm
} % section title spacing

\renewcommand\labelitemi{$\vcenter{\hbox{\small$\bullet$}}$} % custom bullet points
\newenvironment{highlights}{
    \begin{itemize}[
        topsep=0.10 cm,
        parsep=0.10 cm,
        partopsep=0pt,
        itemsep=0pt,
        leftmargin=0 cm + 10pt
    ]
}{
    \end{itemize}
} % new environment for highlights


\newenvironment{highlightsforbulletentries}{
    \begin{itemize}[
        topsep=0.10 cm,
        parsep=0.10 cm,
        partopsep=0pt,
        itemsep=0pt,
        leftmargin=10pt
    ]
}{
    \end{itemize}
} % new environment for highlights for bullet entries

\newenvironment{onecolentry}{
    \begin{adjustwidth}{
        0 cm + 0.00001 cm
    }{
        0 cm + 0.00001 cm
    }
}{
    \end{adjustwidth}
} % new environment for one column entries

\newenvironment{twocolentry}[2][]{
    \onecolentry
    \def\secondColumn{#2}
    \setcolumnwidth{\fill, 4.5 cm}
    \begin{paracol}{2}
}{
    \switchcolumn \raggedleft \secondColumn
    \end{paracol}
    \endonecolentry
} % new environment for two column entries

\newenvironment{threecolentry}[3][]{
    \onecolentry
    \def\thirdColumn{#3}
    \setcolumnwidth{, \fill, 4.5 cm}
    \begin{paracol}{3}
    {\raggedright #2} \switchcolumn
}{
    \switchcolumn \raggedleft \thirdColumn
    \end{paracol}
    \endonecolentry
} % new environment for three column entries

\newenvironment{header}{
    \setlength{\topsep}{0pt}\par\kern\topsep\centering\linespread{1.5}
}{
    \par\kern\topsep
} % new environment for the header

\newcommand{\placelastupdatedtext}{% \placetextbox{<horizontal pos>}{<vertical pos>}{<stuff>}
  \AddToShipoutPictureFG*{% Add <stuff> to current page foreground
    \put(
        \LenToUnit{\paperwidth-2 cm-0 cm+0.05cm},
        \LenToUnit{\paperheight-1.0 cm}
    ){\vtop{{\null}\makebox[0pt][c]{
        \small\color{gray}\textit{Last updated in September 2024}\hspace{\widthof{Last updated in September 2024}}
    }}}%
  }%
}%

\newcommand{\customref}[3][primaryColor]{\href{#2}{\color{#1}{\underline{#3}}}}

% save the original href command in a new command:
\let\hrefWithoutArrow\href

% new command for external links:


\begin{document}
    \newcommand{\AND}{\unskip
        \cleaders\copy\ANDbox\hskip\wd\ANDbox
        \ignorespaces
    }
    \newsavebox\ANDbox
    \sbox\ANDbox{$|$}

    \begin{header}
        \fontsize{25 pt}{25 pt}\selectfont John Cho \\
        \normalsize
        3$^{\text{rd}}$ Year Software Engineering Student at McMaster University
        % \vspace{5 pt}

        \mbox{Hamilton, ON}%
        \kern 5.0 pt%
        \AND%
        \kern 5.0 pt%
        \mbox{\customref{mailto:johnnychox@gmail.com}{johnnychox@gmail.com}}%
        \kern 5.0 pt%
        \AND%
        \kern 5.0 pt%
        \mbox{\customref{tel:+1-226-340-6077}{(226) 340-6077}}%
        \kern 5.0 pt%
        \AND%
        \kern 5.0 pt%
        \mbox{\customref{https://www.linkedin.com/in/john-cho-7411a224b/}{linkedin.com/in/john-cho}}%
        \kern 5.0 pt%
        \AND%
        \kern 5.0 pt%
        \mbox{\customref{https://github.com/chosterto}{github.com/chosterto}}%
    \end{header}

    % \vspace{5 pt}


    \section{Education}



        \fontsize{9 pt}{9 pt}\selectfont
        \begin{twocolentry}{
            Sept 2022 – May 2027
        }
            \textbf{McMaster University}, BS in Software Engineering\end{twocolentry}

        \vspace{0.10 cm}
        \begin{onecolentry}
            \begin{highlights}
                \item \textbf{GPA:} 3.9/4.0
                \item \textbf{Coursework:} Computer Architecture, Discrete Math, Object-Oriented Programming, Data Structures and Algorithms, Linear Optimization, Digital Signals and Systems
            \end{highlights}
        \end{onecolentry}



    
    \section{Extracurriculars}



        
        \begin{twocolentry}{
            Nov 2022 – present
        }
            \textbf{Software Team Lead}, \textbf{\customref[blue]{https://www.marsatmac.ca/}{McMaster Mars Rover Team}} -- Hamilton, ON\end{twocolentry}

        \vspace{0.10 cm}
        \begin{onecolentry}
            \begin{highlights}
                \item Lead a team of 5 people to successfully enhance and maintain current rover software stack using tools such as Git and Kanban boards
                \item Using ROS 2 to develop complex control systems for the rover such as automonous 3D mapping with SLAM, 2D GPS mapping (mapviz), and point-to-point navigation with obstacle avoidance
                \item Developed firmware for sensor boards to communicate GPS, IMU, and temperature data over ethernet using micro-ROS and control custom motor controller boards with PID over CAN bus
                \item Team placed 1$^{\text{st}}$ in Canada (4$^{\text{th}}$ overall) at \customref[blue]{https://circ.cstag.ca/2023/}{\textbf{Summer CIRC 2023}} and 2$^{\text{nd}}$ overall at \customref[blue]{https://circ.cstag.ca/2024-winter/}{\textbf{Winter CIRC 2024}}
            \end{highlights}
        \end{onecolentry}


        \vspace{0.2 cm}

        \begin{twocolentry}{
            Oct 2018 – June 2022
        }
            \textbf{Programming Leader}, \textbf{\customref[blue]{https://www.firstinspires.org/robotics/frc}{FIRST Robotics Competition}} -- LaSalle, ON\end{twocolentry}

        \vspace{0.10 cm}
        \begin{onecolentry}
            \begin{highlights}
                \item Member and leader of the programming section for FRC Team 772, the \href{https://www.sabrerobotics.com/}{Sabre Bytes}
                \item Drive Team Operator for 2021-2022 season, Rapid React
                \item Wrote commands to control drivetrain, flywheel, turret, and intake subsystems of robots using WPILibC++
                \item Mainly responsible for PID and vision control of turret to automatically aim and shoot balls into the Hub using data from motor encoders and limelight cameras
            \end{highlights}
        \end{onecolentry}



    
    \section{Experience}



        
        \begin{twocolentry}{
            July 2023 – Aug 2023
        }
            \textbf{Camp Counselor}, \textbf{\customref[blue]{https://stemcamp.ca/}{STEM Camp}} -- Hamilton, ON\end{twocolentry}

        \vspace{0.10 cm}
        \begin{onecolentry}
            \begin{highlights}
                \item Managed a group of 20 to 30 kids interested in learning STEM
                \item Taught basic programming concepts using MakeCode
                \item Helped campers utilize micro:bit microcontrollers to control motors, servos, pumps, and sensors to complete a variety of tasks such as making automated robots to plant seeds, basic wind turbines, and water plant dispensers
            \end{highlights}
        \end{onecolentry}



    
    \section{Projects}



        
        \begin{onecolentry}
            \textbf{\customref[blue]{https://drive.google.com/drive/folders/1CYcbyKpHjH-DZPU5JNhlTpTpQmKvzFWN}{Minecraft Turing Machine}}\end{onecolentry}

        \vspace{0.10 cm}
        \begin{onecolentry}
            \begin{highlights}
                \item Created a Turing Machine using Minecraft redstone, complete with a functional memory tape and 8-bit register to store current state, write symbol, and tape direction
                \item Program is a FSM (Finite-state machine) with 14 states which accepts the language or set of strings $\mathcal{L} = \left\{a^{n}b^{n}c^{n} \ | \ n \geq 0 \right\}$
                \item Utilized many concepts in digital systems and discrete math such as combinational logic, k-maps, formal language theory, finite automata, and state minimization
            \end{highlights}
        \end{onecolentry}


        \vspace{0.2 cm}

        \begin{onecolentry}
            \textbf{\customref[blue]{https://github.com/chosterto/PurePursuitTracking}{Pure Pursuit Path Controller}}\end{onecolentry}

        \vspace{0.10 cm}
        \begin{onecolentry}
            \begin{highlights}
                \item Implemented a controller of the pure pursuit path following algorithm for a differential drive system
                \item Works by inputting a set of waypoints, then calculates a smooth path for the robot to follow while feeding back constant odometry data to keep track of its position relative to the path
                \item Used by FIRST Robotics Team 772 to successfully traverse and collect balls around the field in autonomous mode (video demonstration \textbf{\customref[blue]{https://drive.google.com/file/d/1KtWkwZrmCklio3peCb4nabCr_LRlsecF/view}{here}})
            \end{highlights}
        \end{onecolentry}


        \vspace{0.2 cm}

        \begin{onecolentry}
            \textbf{\customref[blue]{https://github.com/chosterto/Regression-Analyzer}{Polynomial Regression Calculator}}\end{onecolentry}

        \vspace{0.10 cm}
        \begin{onecolentry}
            \begin{highlights}
                \item Python script that takes in a CSV file of x and y values and outputs a polynomial that best fits the set of points
                \item Calculates coefficients of polynomial equation using matrix operations
                \item Uses Bayesian information criterion to select a desired model and  prevent overfitting the data
                \item Used by FIRST Robotics Team 772 to find relationship between distance of Hub and flywheel speed needed to shoot balls into the Hub
            \end{highlights}
        \end{onecolentry}



    
    \section{Skills}



        
        \begin{onecolentry}
            \textbf{Languages:} C++, C, Java, Python, Verilog, Bash, YAML, UML
        \end{onecolentry}

        \vspace{0.2 cm}

        \begin{onecolentry}
            \textbf{Technologies:} ROS, ROS 2, micro-ROS, Git, CMake, SSH, Arduino IDE, STM32CubeIDE, Linux, Docker, SLAM, 
            PID, SPI, I2C, CAN, UART, SonarQube, Maven
        \end{onecolentry}


    

\end{document}